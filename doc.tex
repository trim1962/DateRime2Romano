\documentclass[10pt,a4paper]{book}
\usepackage[utf8]{inputenc}
\begin{document}
	Il calendario latino o per meglio dire Romano ha avuto nei secoli molte trasformazioni e adattamenti.
	Abbiamo diversi calendari 
	\begin{description}
		\item[Romolo] All'inizio aveva 10 mesi e iniziava a marzo. Era di dieci mesi e terminava a Dicembre e durava 304 giorni. Da Dicembre a Marzo vi era un vuoto in cui non si calcolavano i giorni. 
		\item [Numa] L'anno ha 12 mesi l'anno durava 355 giorni con un mese intercalare (mercedonio) aggiunto ogni due anni. Ogni mese aveva tre date fisse: le calende, le none e le idi. Le calende erano il primo giorno del mese, le none il quinto e le idi il quindicesimo. I rimanenti giorni erano indicati per differenza rispetto a questi giorni.
		\item [Giulio Cesare]  46 d C. Introduce l'anno bisestile (in origine ogni tre anni). Il giorno in più viene introdotto a febbraio inserendo un giorno dopo il 24 febbraio il bis sesto essendo il 24 il sesto prima delle calende di marzo.  L'anno dura 265 giorni e 6 ore.
		\item [44 a.C] Dopo la morte di Cesare il quintile viene chiamato Luglio in onore gi Giulio Cesare.
		\item [8 a.C] viene corretto l'errore e si passa a un giorno bisestile ogni quattro anni. 
		Setile cambia in Agosto in onore di Augusto. 
	\end{description}
	
	
\end{document}